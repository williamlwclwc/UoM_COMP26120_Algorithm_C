\documentclass{article}

\title{COMP26120 Lab 10}
\author{?}

\begin{document}
\maketitle

% PART 1 %%%%%%%%%%%%%%%%%%%%%%%%%%%%%%%%%%%%%%%%%%%%%%%%%%%%%%%%%%%%%%%%%%%%%%

\section{The small-world hypothesis}
\label{sec:small world}
% Here give your statement of the small-world hypothesis and how you
% are going to test it.
I think the small-world hypothesis is true, and I will test for it by checking:
for each 2 people in the graph, try to find a shortest path whose length is less than 6.


\section{Complexity Arguments}
\label{sec:complexity}
% Write down the complexity of Dijkstra's algorithm and of Floyd's algorithm.
% Explain why, for these graphs, Dijkstra's algorithm is more efficient.
Dijkstra using priority queue has the complexity of O\(|E|+|V|log|V|\),
where E is the number of edges, and V is the number of vertexes.
We also need to apply Dijkstra for every pair of vertexes,
which means that the overall complexity would be O\(|V|*\(|E|+|V|log|V|\)\).
Floyd has the complexity of O\(|V|^3\), where V is the number of vertexes.
And we can approximately calculate the scale of time for Caltech.gx and Oklahoma.gx using Dijkstra and Floyd,
Dijkstra: O\(|V|*\(|E|+|V|log|V|\)\)
\(1\) Caltech.gx: 769*\(33312+769*log\(769\)\) = 31,530,538
\(2\) Oklahoma.gx: 17425*\(1785056+17425*log\(17425\)\) = 3,565,906,018
Floyd:  O\(|V|^3\)
\(1\) Caltech.gx: 769^3 = 454,756,609
\(2\) Oklahoma.gx: 17425^3 = 5,290,763,640,625
We can see from the approxiamation calculated using time complexity, Dijkstra algorithm
is much less than Floyd algorithm, so Dijkstra algorithm is more efficient.

% PART 2 %%%%%%%%%%%%%%%%%%%%%%%%%%%%%%%%%%%%%%%%%%%%%%%%%%%%%%%%%%%%%%%%%%%%%%

\section{Part 2 results}
\label{sec:part2}
% Give the results of part two experiments.
Caltech: 0.4s, it is not a "small world".
Oklahoma: 3066.7s, it is not a "small world".


% PART 3 %%%%%%%%%%%%%%%%%%%%%%%%%%%%%%%%%%%%%%%%%%%%%%%%%%%%%%%%%%%%%%%%%%%%%%

\section{Part 3 results}
\label{sec:part3}
% Give the results of part three experiments.


\section{Conclusions}
\label{sec:conclusions}
% Give your conclusions from the above experiments 


\end{document}
